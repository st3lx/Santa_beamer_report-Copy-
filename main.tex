\documentclass[aspectratio=1610]{beamer}
\hypersetup{
        unicode=true,
        linkcolor=blue,
        anchorcolor=blue,
        citecolor=green,
        filecolor=black,
        urlcolor=blue
    }

%%%%% PACKAGES HERE
%% \usepackage{}
\usepackage{amsmath}
\usepackage{amssymb}
\usepackage{listings}
\usepackage[cache=false]{minted}
\usepackage{textcomp}
\usepackage{xcolor}
\usepackage{tikz}
\usetikzlibrary{positioning,calc}
\usepackage{graphicx}
\usepackage{hyperref}
\usepackage{amsmath}
\usepackage{listings}
\usepackage{fontawesome}
\usepackage[english]{babel}

\usepackage[backend=biber, style=numeric]{biblatex} 
\addbibresource{references.bib}

\input{smile_styles}

%% Start from here.
\title{Improvement of the Minkowski Constant\\ 
       in Multidimensional Best Approximation\\}
\author{Rongqing Wang, Zhongyi Li, Tianer Tang\\
        (Santa's Little Helpers)}
\date\today




\begin{document}

\begin{frame}[plain]
  \titlepage
\end{frame}


\begin{frame}{Outline}
  \tableofcontents

  // This frame is the outline page which contains the full outline.
  \end{frame}



\section{Preliminaries}
\subsection{1.1 Definitions of SA, LF, and Exponents of Growth}

\begin{frame}
  \frametitle{1.1.1 Simultaneous Approximation (SA)}
Let $\boldsymbol{\alpha} = (\alpha_1,\dots,\alpha_d) \in \mathbb{R}^d/ \mathbb{Q}^d$ .\\
\emph{Simultaneous approximation} seeks best approximation vectors
\[
(q_\nu, \mathbf{p_\nu}) = (q_\nu, p_{\nu,1}, \dots, p_{\nu,d}) \in \mathbb{Z}^{d+1}, \quad q_\nu > 0,
\]
such that the approximation error $\mathbf{|r_\nu|} =||q_\nu \boldsymbol{\alpha} - \mathbf{p_\nu}||$ equals the irrationality measure function
\[
\psi_\alpha(t) := \min_{\mathbf{p_\nu} \in \mathbb{Z}^d, 1 \le q_\nu \le t} \ ||q_\nu \boldsymbol{\alpha} - \mathbf{p_\nu}||.
\]
By definition, $\psi_\alpha(t) $ strictly decreases for increasing \(q_\nu\), so the \emph{best approximation sequence} $\{(q_\nu, \mathbf{p}_\nu)\}_{\nu \ge 1}$ is obtained by arranging these vectors in order of strictly increasing $q_\nu$.
\end{frame}

\begin{frame}
  \frametitle{1.1.2 Linear Form Approximation (LF)}
Let $\boldsymbol{\alpha} = (\alpha_1,\dots,\alpha_d) \in \mathbb{R}^d$ be totally irrational (i.e. , $1, \alpha_1, \dots, \alpha_d$ are linearly independent over $\mathbb{Q}$).\\
\emph{Linear form approximation} seeks best approximation vectors
\[
(\mathbf{q_\nu}, p_\nu) = (q_{\nu,1}, \dots, q_{\nu,d}, p_\nu) \in \mathbb{Z}^{d+1}, \quad p_\nu > 0,
\]
such that the approximation error
\[
|r_\nu| := \left| \mathbf{q_\nu \boldsymbol{\alpha}} + p_\nu \right|
\]
equals the linear form irrationality measure function
\[
\psi_\alpha^*(t) := \min_{\substack{(p_\nu, \mathbf{q_\nu}) \in \mathbb{Z}^{d+1} \setminus \{\mathbf{0}\} \\ 0 < |\mathbf{q_\nu}| < t}} \left|| \mathbf{q_\nu \boldsymbol{\alpha}} + p_\nu \right||.
\]
The \emph{best approximation sequence} ${{\mathbf{\{q_\nu}, p_\nu}}\}_{\nu \ge 1}$ is obtained by arranging these vectors in order of strictly increasing norm such that $\psi_\alpha^*(t)$ decreases.

\end{frame}

\begin{frame}
  \frametitle{Exponents of Growth}
Let $(q_\nu, r_\nu)$ be the denominators and remainders of the SA best approximation sequence.  
The \emph{denominator growth exponent} in SA is
\[
\liminf_{\nu \to \infty} \ q_\nu ^{1/\nu}.
\]
Similarly, the \emph{remainder decay exponent} in linear form approximation is
\[
\limsup_{\nu \to \infty} \ \delta_\nu ^{1/\nu}
\]

\end{frame}

\subsection{Duality between SA and LF}
\subsection{Outline of Achievements}

\section{Simultaneous Approximation}

\subsection {Proof of Lagarias}
\begin{frame}
  \frametitle{}
\begin{theorem}[Lagarias, 1982, \cite{Lagarias82}]
If \(\alpha\) has only one irrational coordinate, then for all sufficiently large enough \(\alpha\), \[q_{k+2}\geq q_{k+1}+q_k\]
If \(\alpha\) has both coordinates irrational, then \[q_{k+3}<q_{k+1}+q_k\]
can only occur when
\[q_{k+3}=q_{k+2}+q_{k+1}-q_{k}.\]
\end{theorem}
  \begin{corollary}
  For the sup norm \(||\cdot ||_\infty\) on \(R^2\) the minimal growth rate for SA denominators satisfies
      \begin{equation}
    G(|\cdot ||_\infty\ ) = \liminf_{\nu\to\infty} q_{\nu}^{1/\nu}\geq \theta,
\end{equation}
where \(\theta = \sqrt{\phi}\) is the largest real root of
\(\theta^4=\theta^2+1.\)
  \end{corollary}
\end{frame}

\subsection {Proof}
\begin{frame}
  \frametitle{Proof}
Graph.
\end{frame}
\subsection {Proof}
\begin{frame}
  \frametitle{Proof}
Show that\[q_{k+i}-q_{k+j}>q_{k+j+1},(?)\] 
then
\[q_{k+3}\geq q_{k+i}\>q_{k+j}+q_{k+j+1} \]
for \(i,j\in\{0,1,2,3\}\)
\end{frame}



\subsection{Improving the Minkowski Constant by Exponential Growth of {$q_\nu$}}

\section {Linear Form}

\subsection {Sub sec}
\begin{frame}
  \frametitle{Something}
\[\delta_{\nu}+\delta_{\nu-1}\leq \delta_{\nu-k},\]
where $\delta_{\nu}=\|\overline{q}_{\nu}\cdot \overline{\alpha} \|_{\mathbb{Z}^d}$
\end{frame}

\begin{frame}{Frame Title}
    ?Want to prove 
    \[|\overline{q}_{\nu+12}\geq2|\overline{q}_\nu| |\]
\end{frame}

\section{Duality of the Minkowski Constant from SA to LF}




\addtobeamertemplate{frametitle}{}{%
    \vspace*{-1.5em} % 将标题下移,从而压缩标题区域,留出更多内容显示空间
}
  \begin{frame}[allowframebreaks]
    \frametitle{References}
   \printbibliography
\end{frame}



\begin{frame}{}
\begin{center} \Huge Thank You! \end{center} 
\end{frame}
\end{document}